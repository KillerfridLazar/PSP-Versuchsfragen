\documentclass[12pt]{article}
\usepackage[utf8]{inputenc}
\title{Versuch 2 Fragen}
\author{
        Vaargk
}

\begin{document}
\maketitle
Was ist ein Stack?
Was ist ein Programm/Prozess?
Was ist ein Scheduler?
Was ist ein Laufzeitkontext? Wozu dient dieser?
Warum und was ist ein Leerlaufprozess?
Was ist ein kritischer Bereich? Was ist eine atomare Operation?
Was passiert bei einem Stack Overflow? Was bedeutet inkosistenz?
Was für ein Verhalten passiert beim Stack Overflow und kann man ihn rückgängig machen?
Wie erkannt man einen Stack overflow?
Wie ist der SRAM aufgeteilt?
Strings müssen in Flash gespeichert werden sonst speicherknappheit.
Was speichern wir zu einem Prozess?
Wenn man einen neuen Prozess anlegt was und in welcher Reihenfolge müssen wir dann machen?
Wenn der Scheduler wechselt was und in welcher Reihenfolge passiert dann? Ganz kleinschrittig!!!
Was passiert beim Prozessstack vorbbereiten?
Werden die register beim neuen Prozess direkt beschrieben? -nein Was passiert da genau?
Ganz genau alles zu stackpointer und deren Verwendung wissen. Ganz Wichtig versuchdokument seite 17 Punkt 4
Was passiert beim Scheduler init und genau kleinschrittig und Reihenfolge der Schritte wissen!
Wie funktioniert der wechsel zwischen Prozesstacks?
was macht restoreContext?
\end{document}

