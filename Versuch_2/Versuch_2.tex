\documentclass[12pt]{article}
\usepackage[utf8]{inputenc}
\title{Versuch 2 Fragen}
\author{
        Vaargk, Marko

}

\begin{document}
\maketitle

\begin{itemize}
\item Was ist ein Stack?
\item Was ist ein Programm/Prozess?
\item Was ist ein Scheduler?
\item Was ist ein Laufzeitkontext? Wozu dient dieser?
\item Warum und was ist ein Leerlaufprozess?
\item Was ist ein kritischer Bereich? Was ist eine atomarer Bereich?
\item Was passiert bei einem Stack Overflow? Was bedeutet Inkosistenz?
\item Was für ein Verhalten passiert beim Stack Overflow und kann man ihn rückgängig machen?
\item Wie erkannt man einen Stack overflow?
\item Wie ist der SRAM aufgeteilt?
\item Strings müssen in Flash gespeichert werden sonst Speicherknappheit
\item Was speichern wir zu einem Prozess?
\item Wenn man einen neuen Prozess anlegt was und in welcher Reihenfolge müssen wir dann machen?
\item Wenn der Scheduler wechselt was und in welcher Reihenfolge passiert dann? Ganz kleinschrittig!!!
\item Was passiert beim Prozessstack vorbbereiten?
\item Werden die Register beim neuen Prozess direkt beschrieben? -nein Was passiert da genau?
\item Ganz genau alles zu stackpointer und deren Verwendung wissen. Ganz wichtig Versuchsdokument Seite 17 Punkt 4
\item Was passiert beim Scheduler init und genau kleinschrittig und Reihenfolge der Schritte wissen!
\item Wie funktioniert der Wechsel zwischen Prozesstacks?
\item Was macht restoreContext?
\item Was passiert wenn in os_exec die ISR aufgerufen wird? - Nichts da kritischer Bereich
\end{itemize}
\end{document}

